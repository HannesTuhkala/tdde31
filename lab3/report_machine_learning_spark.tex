\documentclass[a4paper,titlepage,12pt]{article}
\usepackage[utf8]{inputenc} %Make sure all UTF8 characters work in the document
\usepackage{listings} %Add code sections
\usepackage{color}
\usepackage{graphicx}
\usepackage{titling}
\usepackage{textcomp}
\usepackage[hyphens]{url}
\usepackage[bottom]{footmisc}
\usepackage[yyyymmdd]{datetime}
\usepackage{fancyhdr}
\usepackage{fancyvrb}
\usepackage{tikz}
\usepackage{enumerate}
\usetikzlibrary{arrows}
\definecolor{listinggray}{gray}{0.9}
\definecolor{lbcolor}{rgb}{0.9,0.9,0.9}

%Set page size
\usepackage{geometry}
\geometry{margin=3cm}
\usepackage{parskip} 

\renewcommand{\dateseparator}{-}
%\renewcommand{\figurename}{Figur}
%\renewcommand{\contentsname}{Innehållsförtäckning}

\RecustomVerbatimCommand{\VerbatimInput}{VerbatimInput}%
{fontsize=\footnotesize,
 %
 frame=lines,
 framesep=2em
}

\DeclareGraphicsExtensions{.ps}
\DeclareGraphicsRule{.ps}{pdf}{.pdf}{`ps2pdf -dEPSCrop -dNOSAFER #1 \noexpand\OutputFile}

\lstset{literate=%
  {å}{{\r{a}}}1
    {ä}{{\"a}}1
	  {ö}{{\"o}}1
	    {Å}{{\r{A}}}1
		  {Ä}{{\"A}}1
		    {Ö}{{\"O}}1
}

%% Headers och Footers
\pagestyle{fancy}
%% v behövs förmodligen inte..
%% \lhead{\includegraphics[scale=0.2]{LinkUniv_sigill_sv.pdf}}

\begin{document}

{\ }\vspace{45mm}

\begin{center}
	\Huge \textbf{Laboration 3}
	\end{center}
	\begin{center}
		\Large Machine Learning with Spark
	\end{center}

	\vspace{250pt}

	\begin{center}
		\begin{tabular}{|*{2}{p{43mm}|}}
			\hline
			\textbf{Name} & \textbf{Liu-ID} \\	\hline
			{Hannes Tuhkala} & {hantu447} \\	\hline
            {Robin Sliwa} & {robsl733} \\ \hline
			\hline
		\end{tabular}
\end{center}
\newpage

\section{Questions}

\textbf{Show that your choice for the kernels’ width is sensible, i.e. it gives more weight to closer points. Discuss why your definition of closeness is reasonable.}

Our width variables are set to values that would favor closer points of data. h\_distance is set to 150 km, h\_date set to 10 days and h\_time set to 4 hours.
Temperature further than 150 km away should not influence the temperature at the location \textit{that} much. (This is assuming that there is a station approximately 150km away from any location we test, which we assume to be true for our location, atleast.)
Dates further than a 10 days away shouldn't influence the forecast either, since it may be different temperatures at that time.
For the time window, we picked 4 hours since around that time should influence the upcoming temperatures more.

\textbf{It is quite likely that the predicted temperatures do not differ much from one another. Do you think that the reason may be that the three Gaussian kernels are independent one of another? If so, propose an improved kernel, e.g. propose an alternative way of combining the three Gaussian kernels described above.}

Yes, that the three Gaussian kernels are independent of each other may be the reason.
For example, if a datapoint has a high value from the time kernel but low values
from the other two kernels, it may give a skewed representation of the predicted
time at that point.
A better solution would be to multiply the kernels instead
of summing them. This gives a better representation in the data if all three kernels are high, then it is most likely that temperature.

\section{Results}
\textbf{Summing the kernels. The data is: Date, Latitude, Longitude, Predicted temperature.}

('2013-07-24 04:00', 58.4274, 14.826, 4.31) \\
('2013-07-24 06:00', 58.4274, 14.826, 4.6) \\
('2013-07-24 08:00', 58.4274, 14.826, 5.17) \\
('2013-07-24 10:00', 58.4274, 14.826, 5.81) \\
('2013-07-24 12:00', 58.4274, 14.826, 6.21) \\
('2013-07-24 14:00', 58.4274, 14.826, 6.24) \\
('2013-07-24 16:00', 58.4274, 14.826, 5.97) \\
('2013-07-24 18:00', 58.4274, 14.826, 5.61) \\
('2013-07-24 20:00', 58.4274, 14.826, 5.33) \\
('2013-07-24 22:00', 58.4274, 14.826, 5.26) \\
('2013-07-24 24:00', 58.4274, 14.826, 5.49) \\

\textbf{Multiplying the kernels instead}

('2013-07-24 04:00', 58.4274, 14.826, 14.59) \\
('2013-07-24 06:00', 58.4274, 14.826, 16.0) \\
('2013-07-24 08:00', 58.4274, 14.826, 17.63) \\
('2013-07-24 10:00', 58.4274, 14.826, 19.16) \\
('2013-07-24 12:00', 58.4274, 14.826, 20.19) \\
('2013-07-24 14:00', 58.4274, 14.826, 20.51) \\
('2013-07-24 16:00', 58.4274, 14.826, 20.06) \\
('2013-07-24 18:00', 58.4274, 14.826, 18.98) \\
('2013-07-24 20:00', 58.4274, 14.826, 17.62) \\
('2013-07-24 22:00', 58.4274, 14.826, 16.35) \\
('2013-07-24 24:00', 58.4274, 14.826, 15.43) \\

\end{document}
