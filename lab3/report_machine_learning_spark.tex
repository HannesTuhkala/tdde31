\documentclass[a4paper,titlepage,12pt]{article}
\usepackage[utf8]{inputenc} %Make sure all UTF8 characters work in the document
\usepackage{listings} %Add code sections
\usepackage{color}
\usepackage{graphicx}
\usepackage{titling}
\usepackage{textcomp}
\usepackage[hyphens]{url}
\usepackage[bottom]{footmisc}
\usepackage[yyyymmdd]{datetime}
\usepackage{fancyhdr}
\usepackage{fancyvrb}
\usepackage{tikz}
\usepackage{enumerate}
\usetikzlibrary{arrows}
\definecolor{listinggray}{gray}{0.9}
\definecolor{lbcolor}{rgb}{0.9,0.9,0.9}

%Set page size
\usepackage{geometry}
\geometry{margin=3cm}
\usepackage{parskip} 

\renewcommand{\dateseparator}{-}
%\renewcommand{\figurename}{Figur}
%\renewcommand{\contentsname}{Innehållsförtäckning}

\RecustomVerbatimCommand{\VerbatimInput}{VerbatimInput}%
{fontsize=\footnotesize,
 %
 frame=lines,
 framesep=2em
}

\DeclareGraphicsExtensions{.ps}
\DeclareGraphicsRule{.ps}{pdf}{.pdf}{`ps2pdf -dEPSCrop -dNOSAFER #1 \noexpand\OutputFile}

\lstset{literate=%
  {å}{{\r{a}}}1
    {ä}{{\"a}}1
	  {ö}{{\"o}}1
	    {Å}{{\r{A}}}1
		  {Ä}{{\"A}}1
		    {Ö}{{\"O}}1
}

%% Headers och Footers
\pagestyle{fancy}
%% v behövs förmodligen inte..
%% \lhead{\includegraphics[scale=0.2]{LinkUniv_sigill_sv.pdf}}

\begin{document}

{\ }\vspace{45mm}

\begin{center}
	\Huge \textbf{Laboration 3}
	\end{center}
	\begin{center}
		\Large Machine Learning with Spark
	\end{center}

	\vspace{250pt}

	\begin{center}
		\begin{tabular}{|*{2}{p{43mm}|}}
			\hline
			\textbf{Name} & \textbf{Liu-ID} \\	\hline
			{Hannes Tuhkala} & {hantu447} \\	\hline
            {Robin Sliwa} & {robsl733} \\ \hline
			\hline
		\end{tabular}
\end{center}
\newpage

\section{Introduction}

\section{Questions}

\textbf{Show that your choice for the kernels’ width is sensible, i.e. it gives more weight to closer points. Discuss why your definition of closeness is reasonable.}

ANSWER:

\textbf{It is quite likely that the predicted temperatures do not differ much from one another. Do you think that the reason may be that the three Gaussian kernels are independent one of another? If so, propose an improved kernel, e.g. propose an alternative way of combining the three Gaussian kernels described above.}

ANSWER:

\section{Results}

\textbf{Using the way as instructed in the lab PM}

('2013-07-04 04:00', 58.4274, 14.826, 4.01)
('2013-07-04 06:00', 58.4274, 14.826, 4.37)
('2013-07-04 08:00', 58.4274, 14.826, 4.86)
('2013-07-04 10:00', 58.4274, 14.826, 5.34)
('2013-07-04 12:00', 58.4274, 14.826, 5.67)
('2013-07-04 14:00', 58.4274, 14.826, 5.78)
('2013-07-04 16:00', 58.4274, 14.826, 5.67)
('2013-07-04 18:00', 58.4274, 14.826, 5.45)
('2013-07-04 20:00', 58.4274, 14.826, 5.2)
('2013-07-04 22:00', 58.4274, 14.826, 5.04)
('2013-07-04 24:00', 58.4274, 14.826, 5.01)

\textbf{Using the new way}

('2013-07-04 04:00', 58.4274, 14.826, 12.96)
('2013-07-04 06:00', 58.4274, 14.826, 13.79)
('2013-07-04 08:00', 58.4274, 14.826, 14.71)
('2013-07-04 10:00', 58.4274, 14.826, 15.55)
('2013-07-04 12:00', 58.4274, 14.826, 16.13)
('2013-07-04 14:00', 58.4274, 14.826, 16.32)
('2013-07-04 16:00', 58.4274, 14.826, 16.12)
('2013-07-04 18:00', 58.4274, 14.826, 15.64)
('2013-07-04 20:00', 58.4274, 14.826, 15.04)
('2013-07-04 22:00', 58.4274, 14.826, 14.43)
('2013-07-04 24:00', 58.4274, 14.826, 13.89)



\end{document}

